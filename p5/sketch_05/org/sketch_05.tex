% Intended LaTeX compiler: xelatex
\documentclass[11pt]{scrartcl}
\setcounter{secnumdepth}{6}
\date{\today}
\title{p5.js sketch 05}
\hypersetup{
 pdfauthor={Alex Lipovka},
 pdftitle={p5.js sketch 05},
 pdfkeywords={},
 pdfsubject={A literate programming approach to p5js coding},
 pdfcreator={Emacs 26.3 (Org mode 9.1.9)}, 
 pdflang={Russian}}
\begin{document}

\maketitle
\setcounter{tocdepth}{6}
\tableofcontents

\section{Введение}
\label{sec:org2e0b15d}

Продолжение цикла программирования на p5.js с помощью \href{https://www.gnu.org/software/emacs/}{emacs} с \href{https://orgmode.org/}{orgmode} в стиле \href{http://www.literateprogramming.com/}{literate programming}.

\href{./sketch\_05.org}{Исходный код данной страницы} и \href{../index.html}{результат}. 

Экспорт исходного кода из данного файла сочетанием клавиш \texttt{C-c C-v C-t}.

В данном примере рассматриваются возможности типографики и получения метрики шрифтов средствами p5.js. В частности, способы подключения разных шрифтов в проект.

В целом можно выделить два принципиально разных подхода:
\begin{enumerate}
\item Использование локальных копий шрифтов (хранение файлов шрифтов на хостинге).
\item Использование онлайн-шрифтов.
\end{enumerate}

\section{Базовый HTML-документ}
\label{sec:org3c8d846}

\begin{minted}[breaklines=true,breakanywhere=true]{html}
<!DOCTYPE html>
<html lang="">

<head>
  <meta charset="utf-8">
  <meta name="viewport" content="width=device-width, initial-scale=1.0">
  <title>p5.js sketch 05</title>
\end{minted}

Добавим ссылку на таблицу стилей.

\begin{minted}[breaklines=true,breakanywhere=true]{html}
<link rel="stylesheet" href="./styles.css">
\end{minted}

А также ссылки на библиотеку p5.js и на исходный код нашей программы.

\begin{minted}[breaklines=true,breakanywhere=true]{html}
<script src="../p5.js"></script>
<script src="./sketch.js"></script>
\end{minted}

И завершение базового файла.

\begin{minted}[breaklines=true,breakanywhere=true]{html}
</head>

<body>
</body>

</html>
\end{minted}

\section{Таблица стилей}
\label{sec:orga35528d}

Таблица стилей будет вынесена в отдельный файл для большей практичности.

Сейчас основная задача — убрать поля и полосы прокрутки, которые возникнут и будт мешать восприятию, если ничего не предпринять.

\begin{minted}[breaklines=true,breakanywhere=true]{css}
html {
    overflow: hidden;
    overflow-x: hidden;
}
body {
    padding: 0;
    margin: 0;
}
canvas {
    display: block;
}
video {
    display: block;
}
\end{minted}

\section{Скетч}
\label{sec:org419c7b8}

\subsection{Настройка области рисования}
\label{sec:org2dd9382}

Настраиваем область рисования и загружаем шрифты.

\label{org7059c9f} \textbf{2-й вариант}. Написание функции, которая будет модифицировать html-документ и формировать ссылки правильного вида для подключения требуемых шрифтов. Данная функция написана на основе \href{https://editor.p5js.org/Roxanne/sketches/r1MCtfFp7}{данной работы}.

\begin{minted}[breaklines=true,breakanywhere=true]{javascript}
function getFontFromGoogle(fontName) {
    let link = document.createElement("link");
    link.id = 'font';
    link.href = "https://fonts.googleapis.com/css?family=" + fontName + '&display=swap';
    console.log(link.href);
    link.rel = "stylesheet";
    document.head.appendChild(link);

    link.addEventListener('load', function () {
        console.log('font ' + fontName + ' loaded');
    });
}
\end{minted}

Подготовим список шрифтов для подключения. Данный список можно расширять, достаточно только знать название требуемого шрифта в библиотеке \href{https://fonts.google.com/}{Google Fonts}.

\begin{minted}[breaklines=true,breakanywhere=true]{javascript}
var fontNames = ['Poiret One', 'Lobster', 'Pacifico', 'Yanone Kaffeesatz', 'Amatic SC'];
\end{minted}

С использованием определенной выше функции подключим шрифты по списку.

\begin{minted}[breaklines=true,breakanywhere=true]{javascript}
function setup() {
    //стандартное создание области рисования размеров с текущее окно
    createCanvas(windowWidth, windowHeight);
    background(255);
    for(let i = 0; i < fontNames.length; i++) {
        getFontFromGoogle(fontNames[i]);
    }
}
\end{minted}

\subsection{Отображение текста}
\label{sec:org25c3baa}

Выведем текст с использованием подключенных шрифтов.

\begin{minted}[breaklines=true,breakanywhere=true]{javascript}
function draw() {
    let pos = new p5.Vector(10, 0);

    background(255);

    for(let i = 0; i < fontNames.length; i++) {
        fill(0);
        textFont(fontNames[i]);
        let fs = map(mouseX, 0, width, 6, 600);
        textSize(fs);
        let str = 'Шрифт — ' + fontNames[i];
//        pos.x = 10;
//        pos.y += 10 + i*1.2*fs + textAscent() + textDescent();
        pos.y += textAscent() + 20;
        text(str, pos.x, pos.y);//, width, textAscent() + textDescent());

        text_width = textWidth(str);
        text_ascent = textAscent();
        text_descent = textDescent();
        text_height = text_ascent + text_descent;
        //метрика всей строки
        noStroke();
        fill(50);
        ellipse(pos.x + 0.5, pos.y + 0.5, 7, 7);
        ellipse(pos.x + text_width + 0.5, pos.y + 0.5, 7, 7);
        stroke(255);
        point(pos.x, pos.y);
        point(pos.x + text_width, pos.y);

        stroke(0, 255, 0);
        line(pos.x, pos.y - text_ascent, pos.x + text_width, pos.y - text_ascent);
        line(pos.x, pos.y + text_descent, pos.x + text_width, pos.y + text_descent);
        pos.y += text_descent;

//        let ch_ext = base_font.getGlyph(str.charAt(i)).topExtent;
    }
}
\end{minted}

\subsection{Обработка изменения размера окна}
\label{sec:org35775f4}

\begin{minted}[breaklines=true,breakanywhere=true]{javascript}
function windowResized() {
    //в случае изменения окна надо перестроить область рисования
    resizeCanvas(windowWidth, windowHeight);
}
\end{minted}
\end{document}
